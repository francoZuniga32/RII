\documentclass[11pt,a4paper]{article}
\usepackage[utf8x]{inputenc}
\usepackage[T1]{fontenc}
\usepackage[spanish]{babel}
\usepackage{amsmath}
\usepackage{amssymb,amsfonts,textcomp}
\usepackage{xcolor}
\usepackage{array}
\usepackage{multirow}
\usepackage{hhline}
\PassOptionsToPackage{hyphens}{url}\usepackage{hyperref}
\usepackage{float}
\usepackage{xkeyval}
\usepackage[pdftex]{graphicx}
\usepackage[yyyymmdd,hhmmss]{datetime}
\usepackage{appendix}
\usepackage{listings}
\usepackage{wasysym}
\definecolor{dkgreen}{rgb}{0,0.6,0}
\definecolor{gray}{rgb}{0.5,0.5,0.5}
\definecolor{mauve}{rgb}{0.58,0,0.82}
\definecolor{lstbackground}{rgb}{0.95,0.95,0.95}
%\lstset{frame=tb,
%	backgroundcolor=\color{lstbackground},
%  language=Bash,
%  aboveskip=3mm,
%  belowskip=3mm,
%  showstringspaces=false,
%  columns=flexible,
%  basicstyle={\small\ttfamily},
%  numbers=none,
%  numberstyle=\tiny\color{gray},
%  keywordstyle=\color{blue},
%  commentstyle=\color{dkgreen},
%  stringstyle=\color{mauve},
%  breaklines=true,
%  breakatwhitespace=true
%  tabsize=3
%}

\lstset{frame=tb,
	backgroundcolor=\color{lstbackground},
%  language=Bash,
  aboveskip=3mm,
  belowskip=3mm,
  showstringspaces=false,
  columns=flexible,
  basicstyle={\small\ttfamily},
  numbers=none,
%  numberstyle=\tiny\color{gray},
%  keywordstyle=\color{blue},
%  commentstyle=\color{dkgreen},
%  stringstyle=\color{mauve},
  breaklines=true,
  breakatwhitespace=true
  tabsize=4
}

\usepackage{verbatim}
\begin{comment}
	\hypersetup{
		pdftex, 
		colorlinks=true, 
		linkcolor=blue, 
		citecolor=blue, 
		filecolor=blue, 
		urlcolor=blue, 
	pdftitle={Software Libre}, 
	pdfauthor={Eduardo Grosclaude}, 
	pdfsubject={Documento de la materia Software Libre}, 
	pdfkeywords={Software Libre, Tecnicatura en Administración de Sistemas y 		Software Libre, Universidad Nacional del Comahue}
	}
\end{comment}	

%\addto\captionsspanish {%
%	\def\appendixname{Apéndices}
%}
% Outline numbering
\setcounter{secnumdepth}{1}
% Reset section numbering between parts
\makeatletter
\@addtoreset{section}{part}
\makeatother  
% List styles
\newcommand\liststyleLi{%
\renewcommand\labelitemi{\tiny${\blacksquare}$}
\renewcommand\labelitemii{\tiny${\square}$}
\renewcommand\labelitemiii{\tiny${\circ}$}
\renewcommand\labelitemiv{\tiny${\circ}$}
}
\newcommand\liststyleLii{%
\renewcommand\labelitemi{{\textbullet}}
\renewcommand\labelitemii{${\circ}$}
\renewcommand\labelitemiii{${\blacksquare}$}
\renewcommand\labelitemiv{{\textbullet}}
}
\newcommand\liststyleLiii{%
\renewcommand\labelitemi{{\textbullet}}
\renewcommand\labelitemii{${\circ}$}
\renewcommand\labelitemiii{${\blacksquare}$}
\renewcommand\labelitemiv{{\textbullet}}
}

\liststyleLi

% Page layout (geometry)
\setlength\voffset{-1in}
\setlength\hoffset{-1in}
\setlength\topmargin{2cm}
\setlength\oddsidemargin{2cm}
\setlength\textheight{23.246668cm}
\setlength\textwidth{17.006cm}
\setlength\footskip{26.144882pt}
\setlength\headheight{1.016cm}
\setlength\headsep{0.508cm}
% Footnote rule
\setlength{\skip\footins}{0.119cm}
\renewcommand\footnoterule{\vspace*{-0.018cm}\setlength\leftskip{0pt}\setlength\rightskip{0pt plus 1fil}\noindent\textcolor{black}{\rule{0.25\columnwidth}{0.018cm}}\vspace*{0.101cm}}
% Pages styles
\makeatletter
\newcommand\ps@Standard{
  \renewcommand\@oddhead{{\raggedleft Cabecera \ } {\raggedright \thepage{}}}
  \renewcommand\@evenhead{\@oddhead}
  \renewcommand\@oddfoot{}
  \renewcommand\@evenfoot{\@oddfoot}
  \renewcommand\thepage{\arabic{page}}
}
\makeatother
% \pagestyle{Standard}
\usepackage{fancyhdr}
\usepackage{sans}

% \renewcommand*\familydefault{\sfdefault}
\pagestyle{fancy}
% footnotes configuration
\makeatletter
\renewcommand\thefootnote{\arabic{footnote}}
\makeatother
\title{Redes II}
\author{Eduardo Grosclaude}
\date{2014-08-13}
\usepackage{graphicx}

\usepackage{xkeyval}
\usepackage{pifont}
\usepackage{xcolor}
\newcommand{\revisar}[1]{{\color{red}[#1]}}



%\usepackage[hundred]{vrsion}

\newcommand{\borrador}{
\revisar{\today, \currenttime  -  Material en preparación, se ruega no imprimir mientras aparezca esta nota}
}


%\newcommand{\nota}[1]{{\color{red}[#1]}}
%\newcommand{\revisar}[1]{}
\newcommand{\nota}[1]{}

\newcommand{\nonota}[1]{#1}

\newcommand{\quotes}[1]{``#1''}

   
\newcommand{\shade}[1]{\textcolor{black!50}{#1}}

% ancho opcional, por defecto 15cm
% \figura{copyleft}{Símbolo de Copyleft}{copyleft.png}
% \figura[6]{copyleft}{Símbolo de Copyleft}{copyleft.png}
\newcommand{\figura}[4][15]{
 \begin{figure}[htbp] 
 \centering 
 \includegraphics[width=#1cm]{./img/#4} 
 \caption{#3} 
 \label{fig:#2} 
 \end{figure} 
}

\newcommand{\tabla}[4]{
 \begin{table} 
 \centering 
 \small
 \begin{tabular}{#3}
 #4
 \end{tabular}
 \caption{#2}
 \label{tab:#1} 
 \end{table} 
}

\usepackage{mdframed}

\newcommand{\recuadro}[1]{
\begin{minipage}[c]{0.84\textwidth}
\begin{mdframed}
#1
\end{mdframed}
\end{minipage}
}

\newcommand{\code}[1]{\lstinline$#1$}
%\newcommand{\code}[1]{\begin{verbatim}{#1}\end{verbatim}}

\hypersetup{colorlinks=true, linkcolor=blue, citecolor=blue, filecolor=blue, urlcolor=blue, 
	pdftitle={Redes II}, 
	pdfauthor={Eduardo Grosclaude}, 
	pdfsubject={Documento de la materia Redes II}, 
	pdfkeywords={Redes, Switching, Alta Disponibilidad, Tecnicatura en Administración de Sistemas y Software Libre, Universidad Nacional del Comahue}}

%TODO Incorporar load balance en LAN con modulo nth de iptables

% --------------------------------------------------------------------
\begin{document}\sloppy
%% \pagestyle{empty}\psset{unit=1in}\begin{pspicture}(8.27,11.69)% use your page size
%   \rput[b](3.5,8){\parbox{5in}{\begin{flushright}\Huge\bfseries\sffamily\color{blue}{Redes II TUASSL}\end{flushright}}}
%   \uput[-90](3.5,8){\color{blue}\rule{5in}{1ex}}
% 	\noindent\includegraphics[width=\paperwidth]{img/Patrik_Goethe_Unsplash.jpg}
% \end{pspicture}
% 



% \input{logofai}
% \logofai{4}{3}{9}{0.3}
% 	%\includegraphics[trim = 10mm 80mm 20mm 5mm, clip, width=3cm]{chick}
% 	%\includegraphics[width=100]{img/Patrik_Goethe_Unsplash.jpg}
% 
% 
% %\usepackage{pdfpages}
% \pagestyle{empty}
% \includegraphics[width=\paperwidth]{img/RII-portada.pdf}
% 
% \begin{minipage}
% {\paperwidth}	
% \includepdf{img/RII-portada.pdf}
% \end{minipage}
% 
% Found the solution here: http://tex.stackexchange.com/questions/5911/how-to-include-pdf-pages-without-a-newpage-before-the-first-page
% 
% the only thing is to add the \section in the pagecommand of \includepdf like this:
% 
% \includepdf[pages=-,scale=.8,pagecommand={\section{title 1}\label{}},linktodoc=true]{myPDF1.pdf}
% \includepdf[pages=-,scale=.8,pagecommand={\section{title 2}\label{}},linktodoc=true]{myPDF2.pdf}

% %\begin{tikzpicture}[remember picture,overlay]
%  \begin{tikzpicture}
%    \node at (current page.center) {\includegraphics[page=1]{img/RII-portada.pdf}};
%  \end{tikzpicture}

% \includepdf[pages=2-last]{MYINCLDOC.pdf} 
% Nada anduvo :(

 :(
\maketitle

\borrador

\abstract {En este escrito se presenta la descripción y material inicial de la asignatura \textbf{Redes II}, para la carrera de Tecnicatura Universitaria en Administración de Sistemas y Software Libre, de la Universidad Nacional del Comahue. 

La materia es electiva, cuatrimestral en modalidad presencial y las clases son de carácter teórico-práctico, desarrolladas en forma colaborativa. Está preparada con los objetivos generales de  \textbf{actualizar y ampliar el instrumental de trabajo en ambientes de redes}. 
  


\newpage
\emph{" "}
\newpage

\tableofcontents

\newpage
\emph{" "}

%----------- P R E S E N T A C I O N  ---------
\newpage
\part {La asignatura}

\input {RII-asignatura}
%
\section{Evaluación}
La evaluación de la materia se realizará mediante trabajos grupales de investigación y desarrollo sobre proyectos de Software Libre, de la siguiente manera.
\begin{itemize}
	\item Los estudiantes se dividirán en grupos de 2 a 5 personas. 
	\item Los grupos desarrollarán trabajos prácticos en etapas que se distribuirán a lo largo de la materia. 
	\item Cada grupo abrirá un diario, blog o wiki de acceso público en cualquier sitio disponible y publicará, mediante el Foro de la materia, la forma de acceder al diario para lectura. Los docentes y los demás estudiantes de la materia podrán acceder al diario del grupo para lectura. Todo cambio en la dirección o forma de acceso deberá ser informado mediante el Foro.
	\item El grupo irá aportando los resultados de cada etapa de los trabajos a su diario, y periódicamente comentará además en clase las experiencias surgidas durante la realización de los trabajos.
	\item El material publicado en el diario será reunido en un documento final que será entregado \textbf{en formato electrónico} al finalizar la materia. El documento indicará tema del trabajo, resumen, integrantes del grupo, desarrollo y conclusiones. 
	\item El documento será acompañado por una presentación de no más de treinta minutos que será expuesta según el cronograma adjunto. 
	\item La acreditación final tendrá en cuenta la calidad del material aportado al diario por el grupo, la calidad de los documentos finales de los trabajos, la presentación oral y la participación en clase ofreciendo la experiencia adquirida durante la realización de los trabajos.
\end{itemize}

\subsection {Trabajo I - Colaboración con proyectos libres}
\subsubsection{Etapa 1}  
Descargar e instalar software ofrecido por un proyecto de Software Libre que esté en actividad (puede tratarse de un entorno de escritorio, un programa de sistema, programas de usuario final, una distribución completa, etc.). Familiarizarse con el software utilizándolo. 
\subsubsection{Etapa 2} 
Basándose en el conocimiento adquirido con el uso del software, colaborar de alguna forma con el proyecto que lo origina: 
\begin{itemize}
	\item traduciendo o localizando parte del software,
	\item generando documentación faltante, 
	\item traduciendo parte de la documentación, 
	\item detectando y denunciando errores en el software o en la documentación,
	\item aportando, modificando o corrigiendo código,
	\item aportando conocimiento a los usuarios del proyecto en blogs, salas de chat, bases de conocimiento, etc.
\end{itemize}
Puede abordarse cualquier cantidad manejable de proyectos. La colaboración debe consistir en alguna interacción positiva y completa con cada proyecto. El grupo incorporará al diario los reportes que acrediten esa interacción. Cuando no sea posible realizar o completar la interacción se indicarán las causas, y las acciones realizadas.

El aporte al proyecto debe efectuarse por los canales establecidos por el proyecto. Si se trata de documentación, respetar el formato utilizado; si es el reporte de un error, hacerlo por la vía preferida por el proyecto, etc.

\subsubsection{Etapa 3} 
El grupo entregará un documento conteniendo la historia de las interacciones con cada proyecto, adjuntando las pruebas en anexos y ofrecerá una presentación.

\subsection {Trabajo II - Evaluación de proyectos libres}

\subsubsection{Etapa 1} 
El grupo enunciará un determinado requerimiento concreto de software que puede ser presentado por un empleador. Algunos ejemplos posibles son:
\begin{itemize}
	\item \quotes{un servidor de correo electrónico que maneje listas},
	\item  \quotes{una aplicación de control de asistencia para empleados},
	\item  \quotes{un sistema de edición de textos para traductores},
	\item  \quotes{un sistema de gestión de contenidos web que incluya workflow}, 
	\item \quotes{un motor de juegos 2D para crear juegos que asistan en la enseñanza de matemática},
	\item  \quotes{un programa de simulación de ataques para evaluar postura de seguridad}, 
	\item \quotes{un sistema de control de stock para zapaterías},
	\item \quotes{una distribución de GNU/Linux para escuelas de arte},
	\item \quotes{una distribución para sistemas empotrados}, etc.
\end{itemize}
El grupo debe comprender el propósito del software requerido y debe contar con al menos un integrante con conocimiento razonable de la temática involucrada. El grupo escribirá una entrada en el diario consignando toda la información posible sobre los requerimientos. 

\subsubsection{Etapa 2} 

\begin{itemize}
	\item El grupo $n$ (en adelante \quotes{el proveedor}) tomará a su cargo el requerimiento del grupo $n+1$ (en adelante \quotes{el cliente}), y se atendrá a dicha descripción para el resto del trabajo. 
	\item El grupo proveedor buscará proyectos de SL que apunten a cubrir esos requerimientos y seleccionará al menos dos proyectos, idealmente tres, de entre ellos.
\end{itemize}

\subsubsection{Etapa 3}
Los proyectos serán comparados en función de varios parámetros o dimensiones.
\begin{itemize}
	\item  ajuste a los requerimientos (actual, previsto o potencial),
	\item  licenciamiento, 
	\item  motivación del desarrollo, 
	\item  modelos de negocio del proyecto, 
	\item  tamaño y permanencia de la comunidad,
	\item  dinámica de soporte, 
	\item  dinámica de actualizaciones y mejoras del software.
\end{itemize}

Se podrán agregar a la comparación uno o más desarrollos no libres. 

Las dudas sobre detalles de los requerimientos serán dirigidas al grupo cliente, y contestadas por aquél, mediante el Foro de la página de la materia.  
\subsubsection{Etapa 4} 
El grupo entregará un documento conteniendo la comparación y haciendo una recomendación final, explicando sus fundamentos. Deberán volcar en el trabajo lo que se vaya aprendiendo durante el curso de la materia, en cada uno de los parámetros o dimensiones nombrados. Finalmente ofrecerán una presentación sobre el trabajo.

\label{sub:acreditacion}

\section {Cronograma}
\begin{tabular}{c|c|l|l|l}
Fecha & Semana & Unidad & Trabajo I & Trabajo II\\
\hline
\hline
11/8 & 1	& 	1. Switching 					& 	& \\
18/8 & 2 	& 								 	& 	& \\
25/8 & 3	& 									&	& \\
1/9 & 4 	& 									&	& \\
\hline
\hline
8/9 & 5		& 	2. Redes Privadas Virtuales 	& 	& \\
15/9 & 6	& 									&	& \\
22/9 & 7 	& 									&   & \\
29/9 & 8	& 									&  	& \\ 
6/10& 9		& 	Parcial I						& 	& \\
\hline
\hline
13/10 & 10	& 	3. Balance de Carga en Redes	& 	& \\ 
20/10 & 11	& 									& 	& \\
27/10 & 12	& 									&	& \\
3/11 & 13	& 									&	& \\
\hline
\hline
10/11 & 14	& 	4. Alta Disponibilidad en Redes	&	& \\
17/11 & 15	& 									&	& \\
24/11 & 16	& 	Parcial II						&	& \\ 
\hline
\end{tabular}



% \begin{tabular}{|r|c|c|c|c|c|c|c|c|}
% \hline
%\textsf{7} & fbox {algo} & & & & & & &\\ 
%\hline
%\textsf{7} & & & & & & & &\\ 
%\hline
%\end{tabular}

% subsection  (end)


%----------- M A T E R I A L ---------
\newpage
\part {Switching}
\input {RII-Switching}


\section{VLANs}
Un diseño clásico de redes de campus consiste en un router que proyecta segmentos de LAN como radios de una estrella (Fig. \ref{fig:bbcolap}). La función de comunicar los radios, que anteriormente era cumplida por el backbone de la red, en este diseño se logra por la conmutación efectuada por el router, por lo cual suele llamarse de backbone colapsado. 

Con este diseño, los dominios de broadcast, y por lo tanto los espacios IP definidos sobre ellos, quedan limitados geográficamente. Si en una zona del campus donde llega un radio de la estrella se necesita ubicar nodos sobre dos dominios de broadcast (porque se desea aislarlos por motivos de seguridad, porque se desea situar equipos sobre dos espacios IP diferentes, o porque se desea limitar la competencia de ambas clases de tráfico por el medio), debe haber dos cableados y deben ocuparse dos bocas del router central. 

\figura[8]{bbcolap}{Backbone colapsado}{backbone-colapsado.jpg}

\figura[12]{vlans}{VLANs}{vlans.jpg}


Con la funcionalidad avanzada de VLANs provista por algunos switches (y definida en el estándar IEEE 802.1Q), el mismo cableado, y el sistema de switches de llegada, puede usarse para conducir dos o más dominios de broadcast. 
 
\begin{itemize}


	\item ¿En qué consiste el diseño de red de campus de backbone colapsado? ¿Qué impacto tiene este diseño sobre la posibilidad de distribuir equipos de diferentes clases sobre una misma región de la red?
	\item 
¿Cuál es la finalidad de definir VLANs en un switch? 

	\item 
¿Cómo se modifica el formato de frame Ethernet para lograr la capacidad de separar los dominios de broadcast al definir VLANs?

	\item ¿Qué debe hacer un switch con VLANs definidas al recibir un frame de broadcast sobre una de sus interfaces? ¿Qué debe hacer con un frame unicast?

	\item En la topología de la figura \ref{fig:vlans}, los tres switches tienen definidas dos VLANs. Los hosts H1, H2 y H4 pertenecen a la VLAN 1, y los hosts H3, H5 y H6 a la VLAN 2. 

	\begin{itemize}
		\item ¿Qué deben hacer los switches con un frame de broadcast recibido desde el host H2?
		\item ¿Qué deben hacer los switches con un frame unicast recibido desde el host H5 y dirigido a H6? ¿Lo mismo, pero desde H5 a H3? ¿Qué diferencia en el formato de los frames existe entre un caso y otro, en cada punto del camino?
		\item ¿Qué condición deben cumplir los ports que interconectan los switches entre sí para poder distribuir las VLANs por toda la topología?
		\item ¿Es posible que una aplicación en el host H5 inicie conexión TCP/IP con una aplicación servidora situada en H2? 
	\end{itemize} 

\item ¿Qué elemento externo es necesario para conectar diferentes VLANs en una misma jerarquía de switches? 
\item ¿Qué son los switches multicapa o \emph{multilayer}? 
\end{itemize} 

% subsection  (end) VLANs


\section {Estudio de caso I}


Una organización desarrolla sus actividades en un campus con dos edificios principales, que alojan las oficinas y talleres. En la organización existen tres áreas principales: Operaciones, Comercialización, e Ingeniería. 

Cada área utiliza un servidor de base de datos principal que es propio del área. La organización cuenta además con un servidor de web y de correo electrónico, un servidor de archivos y un servidor de backups, los tres de uso general para las tres áreas. Se desea que todos los puestos de trabajo puedan acceder además a Internet. 

Una preocupación especial de la organización es darle protección a los servidores y puestos de trabajo de Ingeniería, que no deben ser accedidos desde las demás áreas.
 
La planta del campus y sus edificios, con los principales puntos donde se necesita conectividad, es como indica la Fig. \ref{fig:caso01}. En este diagrama se muestran, con diferentes colores, los puestos de trabajo de cada área. 


\figura[12]{caso01}{Conectando el campus de una organización}{caso01.jpg}


En base a esta información, indique:
\begin{itemize}
	\item Dónde situaría los servidores mencionados.
	\item Dónde ubicaría los elementos de conectividad (switches y routers).
	\item Qué cantidad de puertos debería tener cada uno de estos elementos.
	\item Con qué medios (cobre, fibra, inalámbricos) vincularía los clientes y los elementos de conectividad.
	\item Cómo distribuiría direcciones IP para los clientes y servidores. 
	\item Si utilizaría alguna arquitectura de VLANs, cuál, y por qué. En caso afirmativo, cómo relacionaría las VLANs entre sí, qué enlaces entre switches y routers deben ser de trunking y por qué.
	\item Si utilizaría alguna forma de redundancia, y en caso afirmativo, cuál debe ser el camino normal del tráfico y por qué.
\end{itemize}








\newpage
\part {Redes Privadas Virtuales}

\section{Elementos de las VPN}
\begin{itemize}
	\item VPN = \emph{Virtual Private Network}
	\item Un software específico (cliente-servidor) establece una red \emph{virtual} entre las entidades
	\item Sólo aquellos miembros autorizados de la red virtual \emph{privada} pueden utilizarla
	\item \emph{Tunneling} o encapsulamiento de ciertas capas en otras
	\item Encriptación
\end{itemize}


\subsection{Motivación de las VPN}
\begin{itemize}
	\item El software de la organización evoluciona a aplicaciones colaborativas (Groupware, CRM , ERP...)
	\item Trabajo de todos los colaboradores sobre el espacio digital de la organización
	\item Diferentes sedes, o colaboradores (domiciliarios o móviles), en diferentes lugares de la región o del mundo
	\item Soporte universal de Internet en lugar de enlaces arrendados, dedicados
	\item Requerimientos de seguridad
	\item Firewalls y NAT requieren \emph{conduits} o perforaciones para acceder a los servicios de las sedes
 \end{itemize}



\subsection{Modelos de VPN}
\begin{itemize}
	\item Implementadas y administradas por un proveedor de comunicaciones
	\item Implementadas y administradas por el usuario sobre un transporte común (Internet)
\end{itemize}


\subsection{Niveles de implementación}
\begin{itemize}	 

	\item De nivel de Enlace o de capa 2

	\begin{itemize}
		\item Único espacio IP y dominio de broadcast
		\item La VPN trafica frames
		\item Bridging
	\end{itemize} 

	\item De nivel de Red o de capa 3

	\begin{itemize}
		\item  Varias redes con diferentes espacios IP y un servidor VPN que las conecta como backbone 
		\item La VPN trafica paquetes
		\item Ruteo
	\end{itemize}
\end{itemize}


\subsection{Algunas tecnologías}
\begin{itemize}
	\item GRE (RFCs 1701, 1702) Mecanismo general de tunneling, no define encriptación
	\item Protocolos de capa 2: PPTP (Microsoft), L2F, L2TP, L2Sec
	\item Protocolos de capa 3: IPSec
	\item Protocolos de capa 4: SSL/TLS, OpenVPN
\end{itemize}


\figura{vpnbrid} {VPN de capa 2, equivalente a un bridge} {vpn1-bridge.jpg}

\figura{vpnrut} {VPN de capa 3, equivalente a un router} {vpn1-router.jpg}


\section{OpenVPN}

\begin{itemize}
	\item OpenVPN puede funcionar como VPN de capa 2 o de capa 3. 
	\item Proyecto libre compatible con Windows.
	\item No interoperable con IPSec.
	\item Dos procesos, cliente y servidor, corren en diferentes hosts en diferentes redes. 
	\item Cliente y servidor establecen una conexión TCP/IP que utilizan como túnel.
	\item El resto del trabajo lo hace el bridging (en las VPN de capa 2) o el ruteo (de capa 3). 
	\item La arquitectura de OpenVPN (y otras implementaciones de redes virtuales) es un ejemplo claro de uso de interfaces virtuales como los bridges y tun/tap. 
\end{itemize}

\subsection{Soporte necesario para OpenVPN}
\begin{description}
	\item [Tun/Tap] Dispositivos de red virtuales. Trafican unidades de datos entre procesos. 
	\begin{itemize}
		\item Un dispositivo tun (de capa 3) trafica paquetes, mientras que un tap (de capa 2) intercambia frames. 
		\item El propósito básico de Tun/Tap es la creación de túneles.
		\item Un proceso que tiene un dispositivo tun abierto puede escribir un paquete en él.  Del mismo modo, un dispositivo tun que recibe un paquete IP lo envía a un proceso de usuario que tiene abierto ese dispositivo.
		\item Un proceso que tiene un dispositivo tap abierto puede escribir un frame en él.  Del mismo modo, un dispositivo tap que recibe un frame lo envía a un proceso de usuario que tiene abierto ese dispositivo.
	\end{itemize}
	\item [Bridge] Un dispositivo bridge de software funciona exactamente igual que un bridge físico pero entre interfaces del mismo host. Es decir, copia frames de una a otra interfaz de capa 2 (reales o virtuales) efectuando filtrado por dirección MAC e inundando los broadcasts.
	\begin{itemize}
		\item Un bridge Linux \emph{esclaviza} a dos o más interfaces (reales o virtuales).
		\item Al ser incorporada una interfaz a un bridge, pierde su dirección IP y ésta pasa al bridge. 
		\item Los bridges Linux se controlan desde la línea de comandos con el comando brctl.
\end{itemize}
	\item[Ruteo] El kernel utiliza la tabla de ruteo para mover paquetes entre interfaces. Cuando aparece por una interfaz (real o virtual) un paquete de capa 3, se consulta la tabla de ruteo, que dice en cuál interfaz (real o virtual) debe copiarse.  
	\item [OpenSSL] Biblioteca de encriptación de uso general, para proteger la privacidad del tráfico a través de la VPN.
	\item [LZO] Biblioteca de compresión de datos para mejorar la performance del túnel creado por la VPN.
\end{description}



\subsection{Funcionamiento de OpenVPN en capa 2}

\begin{enumerate}
	\item En el host cliente:
	\begin{itemize}
		\item Se crea un dispositivo tapX.
		\item Se establece un bridge entre una interfaz local (ethX) y el tapX.
		\item La dirección IP de la interfaz local pasa al bridge.
		\item El proceso cliente de OpenVPN abre el dispositivo tapX y además establece conexión con el servidor a través de Internet.
		\item Cuando el host cliente reciba de su red local un frame por ethX, el bridge filtrará el frame hacia el tapX. El proceso cliente de OpenVPN recibirá el frame y lo derivará por la conexión TCP/IP hacia el servidor.
		\item Cuando el proceso cliente reciba un frame a través de la conexión TCP/IP, lo escribirá por la interfaz tapX. El bridge del host cliente recibirá el frame y le aplicará la acción de bridging correspondiente. 
	\end{itemize}
	\item En el host servidor:
	\begin{itemize}
		\item Se crea un dispositivo tapX.
		\item Se establece un bridge entre una interfaz local (ethX) y el tapX.
		\item La dirección IP de la interfaz local pasa al bridge.
		\item El proceso servidor de OpenVPN abre el dispositivo tapX y espera pedido de conexión del cliente a través de Internet.
		\item Cuando el host servidor reciba de su red local un frame por ethX, el bridge filtrará el frame hacia el tapX. El proceso servidor de OpenVPN lo tomará y lo derivará por la conexión TCP/IP hacia el cliente.
		\item Cuando el proceso servidor reciba un frame a través de la conexión TCP/IP, lo escribirá por la interfaz tapX. El bridge del host cliente recibirá el paquete y le aplicará la acción de bridging correspondiente. 
	\end{itemize}
\end{enumerate}

\subsection{Funcionamiento de OpenVPN en capa 3}

\begin{enumerate}
	\item En el host cliente:
	\begin{itemize}
		\item Se crea un dispositivo tunX.
		\item El proceso cliente de OpenVPN abre el dispositivo tunX y además establece conexión con el servidor a través de Internet.
		\item El tunX local recibe dirección IP A. Queda establecido un túnel entre procesos cliente y servidor cuyos extremos tienen direcciones IP A y B.
		\item Se establece una ruta que dirige los paquetes destinados a la red Y via tunX (próximo salto: B).
		\item Cuando el host cliente reciba de su red local, o de procesos locales, un paquete destinado a la red Y, lo escribirá en la interfaz de red virtual tunX. El proceso cliente de OpenVPN lo recibirá y lo derivará por la conexión TCP/IP hacia el servidor.
		\item Cuando el proceso cliente reciba un paquete a través de la conexión TCP/IP, lo escribirá por la interfaz tunX. El kernel del host cliente recibirá el paquete y le aplicará las reglas de ruteo correspondientes. 
	\end{itemize}
	\item En el host servidor:
	\begin{itemize}
		\item Se crea un dispositivo tunX.
		\item El proceso servidor de OpenVPN abre el dispositivo tunX y espera pedido de conexión del cliente a través de Internet.
		\item El tunX local recibe dirección IP B. Queda establecido un túnel entre procesos cliente y servidor cuyos extremos tienen direcciones IP A y B. 
		\item Se establece una ruta que dirige los paquetes destinados a la red X via tunX (próximo salto: A).
		\item Cuando el kernel del host servidor reciba de su red local, o de procesos locales, un paquete destinado a la red X, lo escribirá en la interfaz de red virtual tunX. El proceso servidor de OpenVPN tomará el paquete y lo derivará por la conexión TCP/IP hacia el cliente.
		\item Cuando el proceso servidor reciba un paquete a través de la conexión TCP/IP, lo escribirá por la interfaz tunX. El kernel del host servidor recibirá el paquete y le aplicará las reglas de ruteo correspondientes. 
	\end{itemize}
\end{enumerate}

\figura{vpnl2} {VPN de capa 2, trafica frames} {vpn-l2.jpg}

\figura{vpnl3} {VPN de capa 3, trafica paquetes} {vpn-l3.jpg}

\section{Configuración de OpenVPN}

\subsection{Autenticación mediante clave privada}
\begin{itemize}
	\item PKI = Public Key Infrastructure
	\item La seguridad en redes comprende problemas como la confidencialidad y la integridad de los mensajes, y la autenticación de los usuarios o dispositivos
	\item Clave simétrica, con secreto compartido, vs. Claves asimétricas
	\item Dificultad de la distribución de claves, solución: claves asimétricas
	\item Cada usuario tiene $K+$ = Clave pública, $K-$ = Clave privada
	\item Dos escenarios de uso:
	\begin{itemize}
	\item Cuando A encripta con la $K+$ de B:
	\begin{itemize}
		\item A envía a B en forma segura, nadie puede leer el mensaje
		\item Sólo B puede desencriptarlo con su $K-$
		\item Se asegura la confidencialidad
	\end{itemize}
	\item Cuando A encripta con su propia $K-$:
	\begin{itemize}
		\item Cualquiera puede leer el mensaje con la $K+$ de A
		\item Pero sólo A puede haberlo escrito, con su $K-$ (firma digital)
		\item Se aseguran la integridad y la autenticidad
	\end{itemize} 
	\end{itemize}
	\item El problema al utilizar la $K+$ de A que hemos recibido es \emph{asegurar que es realmente la de A}
	\item Se resuelve mediante autoridades de certificación (CA), entidades confiables
	\item La clave pública de A, junto con la identidad de A, firmada digitalmente por una CA, es un \emph{certificado} de A
	\item Conexión segurizada por PKI
	\begin{enumerate}
		\item Fase de autenticación mutua mediante intercambio de certificados, validando los certificados según la CA.
		\item Usando las $K+$ recibidas, las partes acuerdan en forma segura una clave simétrica que sirve para esta sesión.
		\item El resto de la comunicación se encripta usando esta clave simétrica (métodos DES, 3DES, AES).
	\end{enumerate} 
\end{itemize}


\subsection{Crear una Autoridad de Certificación}

En el servidor de VPN, copiar el directorio \lstinline{/usr/share/doc/openvpn/examples/easy-rsa} y todos sus contenidos a /etc/openvpn
\begin{lstlisting}
cd /etc/openvpn/easy-rsa/2.0
vi vars
. vars
./clean-all
./build-dh
./build-ca
./build-key-server servidor
./build-key cliente1
./build-key cliente2
# Migrar los certificados
# En el server:
# ca.crt ca.key dh1024.pem r3.crt r3.key
# 
# En el cliente:
# ca.crt r1.crt r1.key
\end{lstlisting}


\subsection{Documentación online}
\begin{itemize}
	\item Proyecto Lihuén\footnote{\url{http://lihuen.linti.unlp.edu.ar/index.php?title=Configurando_Redes_Privadas_Virtuales_con_OpenVPN}}
	\item OpenVPN HOWTO\footnote{\url{http://openvpn.net/index.php/open-source/documentation/howto.html}}
\end{itemize}


\subsection{Vinculación de LANs mediante Openvpn}
\begin{itemize}
	\item Instalar OpenVPN en los nodos de frontera de las LANs.
	\item Decidir cuál nodo será el servidor. Debe tener dirección IP pública accesible desde los clientes. No es necesaria la traducción DNS salvo que se trate de una IP dinámica.
	\item Preparar una autoridad de certificación (por ejemplo, en el servidor), y certificados para el servidor y para los clientes.
	\item Se pueden utilizar los modelos de archivos de configuración existentes en \url{/usr/share/doc/openvpn/examples}.
	\item En el directorio \lstinline{/etc/openvpn} del servidor preparar el archivo de configuración \lstinline{server.conf}, y en los clientes, \lstinline{client.conf}. Al arrancar, OpenVPN lee todos los archivos con extensión .conf que existan en ese directorio, y crea un proceso por cada uno, con la configuración que contengan.
	\item Modificar los parámetros de configuración:
	\begin{enumerate}
		\item Cliente o servidor
		\item Dirección y port, nodo local o remoto
		\item Protocolo UDP o TCP
		\item Device Tun o Tap, según la capa de operación de la VPN  
		\item Nombres de los archivos de clave y certificado 
		\item Archivo de log, útil para debugging de la configuración y operación
		\item Rutas a redes propias que deban ser inyectadas en el peer
\end{enumerate} 
	\item Arrancar o detener la VPN con el comando \lstinline{/etc/init.d/openvpn [start|stop]}.
\end{itemize}

\subsubsection {OpenVPN de nivel de Red}
Al configurar OpenVPN en capa 3, o nivel de Red, la configuración de OpenVPN puede asumir la creación de rutas entre las redes clientes y las de detrás del servidor. Estas rutas aparecen y desaparecen en la tabla de ruteo de los nodos extremos de la VPN según se activa o desactiva el proceso de OpenVPN.

\begin{itemize}
	\item Para que el cliente conozca las redes detrás del servidor, éste le inyecta las rutas correspondientes con la opción push de la configuración.
	\item El servidor instala rutas a las redes detrás de los clientes si se  especifican con la directiva route de la configuración del servidor. Además, para cada cliente, debe haber un archivo con la directiva iroute dentro del subdirectorio /etc/openvpn/ccd.
\end{itemize}

\subsubsection {OpenVPN de nivel de Enlace}
Al configurar OpenVPN en capa 2, o nivel de Enlace, la configuración debe incluir la administración del bridge entre la interfaz de red local y el dispositivo Tap que es utilizado por la VPN. Los frames que llegan a la interfaz de la red local serán copiados por el bridge en el Tap y seguirán su camino a través de la VPN.

Para la administración del bridge hay dos estrategias básicas: 
\begin{itemize}
	\item El bridge puede tener una definición estática, y ser siempre activado cada vez que arranca el equipo, y sólo ser desactivado cuando se detiene el equipo.
	\item El bridge puede ser activado y desactivado cuando se inicia y detiene OpenVPN. 
\end{itemize}


En el primer caso, la existencia del bridge es completamente independiente de la activación de la VPN. Para esta opción existe una forma de configuración del bridge que es dependiente del sistema operativo (ver Anexo \ref{subsec:staticbridge}). 

Para la segunda opción, es conveniente utilizar los scripts de creación de tap y montado de bridges que se muestran en el Anexo \ref{sec:bridgeupdown} y dispararlos con las directivas up y down de la configuración de OpenVPN. Por ejemplo:

\begin{lstlisting}
...
ca ca.crt
cert server.crt
key server.key
dh dh1024.pem
...
up /etc/openvpn/bridge-up
down /etc/openvpn/bridge-down
\end{lstlisting}


\section{Temas de práctica}
El laboratorio de la Fig. \ref{fig:vpn-lab1} se encuentra implementado sobre máquinas virtuales. 
\begin{comment}Utilizando una consola, encontrará en el directorio \lstinline{labs/openvpn-1} una configuración de laboratorio que se arranca con el comando \lstinline{lstart} y se detiene con \lstinline{lhalt}. 
\end{comment}


\figura{vpn-lab1}{Laboratorio 1 OpenVPN}{vpn-lab1.jpg}

El equipo virtual $R2$, que es gateway default de los otros routers, simula la Internet, en el sentido de que descarta todo tráfico dirigido a redes privadas. Por este motivo las redes con prefijo 10.0.0.0/8, del laboratorio, no son accesibles una desde la otra, siendo necesaria una solución de Red Privada Virtual. La configuración de este equipo $R2$ no debe ser modificada.

\begin{itemize}
	\item Verifique que las direcciones, ruteo default y topología corresponden a la figura.
	\item Verifique servicio de nombres, ping y traceroute a www.google.com.
	\item Ping a los routers de la topología.
	\item Ping a los hosts de la misma LAN.
	\item Ping a los hosts de la LAN opuesta. 
	\item Compruebe navegación en ambiente de caracteres con el comando \code{lynx http://www.google.com}.
	\item Compruebe que puede correr el web server Apache en cualquiera de las PCs. 
	\item Compruebe navegación en ambiente de caracteres desde una PC al servidor Apache del otro nodo de la misma LAN.
\end{itemize}

Una vez que esté familiarizado con el escenario:
\begin{enumerate}
	\item Implementar una VPN de capa 3 entre ambas LANs. Verifique las interfaces existentes en los nodos cabecera y el contenido de sus tablas de ruteo. Comprobar que los  clientes de una de las redes pueden utilizar recursos (ssh, servidor http) de clientes de la otra. Instale servicios basados en broadcast (como Samba, sobre el protocolo SMB) y observe desde qué lugares pueden accederse.
	\item Modificar el direccionamiento IP del laboratorio para que todos los clientes reciban direcciones en la misma subred e implementar una VPN de capa 2. Comprobar que ambas redes forman un único dominio de broadcast. Verifique el comportamiento de los protocolos basados en broadcast como SMB. 
	\item ¿Usaría transporte UDP o TCP para una VPN en capa 2?
\end{enumerate}



\newpage
\part {Balance de carga  y Alta Disponibilidad  en redes}
\section{Bonding}


Acoplamiento de dos o más interfaces de red, creando una interfaz virtual capaz de funcionar en diferentes modos. Se conoce con diferentes nombres (\textit{channel bonding, teaming, link aggregation}). 
Los modos de configuración de bonding definen el comportamiento del conjunto de interfaces y cumplen con diferentes objetivos. Algunos modos proporcionan tolerancia a fallos mediante redundancia; otros aumentan el ancho de banda disponible por agregación de enlaces.

Los modos active-backup, balance-tlb, y balance-alb no requieren una configuración especial en los switches a los cuales está conectado el bond.
Sin embargo, los modos 802.3ad, balance-rr, balance-xor y broadcast requieren capacidades especiales del switch definidas en documentos IEEE.


\begin{description}
\item [Modo 0 (balance-rr)]
Este modo transmite frames en orden secuencial desde el primer esclavo disponible hasta el último. Si un bond tiene dos interfaces reales, y llegan simultáneamente dos frames a ser enviados desde la interfaz bond, el primero será transmitido por el primer esclavo; el segundo frame, por el segundo esclavo; el tercer frame será transmitido por el primer esclavo, etc. Esto provee a la vez balance de carga y tolerancia a fallos. 

\item [Modo 1 (active-backup)]
Este modo coloca a las interfaces esclavas en estado de backup, y sólo se activará una de ellas si se pierde el link de la interfaz activa. En este modo, sólo hay un esclavo activo en el bond en cada momento. Sólo se activa un esclavo diferente si falla el esclavo activo. Este modo provee tolerancia a fallos.

\item [Modo 2 (balance-xor)]
Transmite basándose en una fórmula XOR. Se computa la operación XOR entre la dirección MAC de origen y la de destino, módulo la cantidad de esclavos (es decir, se toma el resto de dividir por la cantidad de esclavos). Este procedimiento tiene el efecto de seleccionar siempre el mismo esclavo para cada dirección MAC destino. Provee balance de carga y tolerancia a fallos.

\item [Modo 3 (broadcast)]
Este modo transmite todos los frames por todas las interfaces esclavas. Es el menos usado, sólo para propósitos específicos, y sólo provee tolerancia a fallos. 

\item [Modo 4 (802.3ad)]
Este modo se conoce como el modo de agregación dinámica de enlaces (Dynamic Link Aggregation). Crea grupos de agregación que comparten la misma velocidad y modos de duplexing. Este modo requiere un switch que soporte la norma IEEE 802.3ad (Dynamic Link).

\item [Modo 5 (balance-tlb)]
Llamado balance de carga adaptativo en transmisión. El tráfico de salida se distribuye de acuerdo a la carga actual y es encolado en cada interfaz esclava. El tráfico entrante es recibido por el esclavo actual. 

\item [Modo 6 (balance-alb)]
Este modo es el de balance de carga adaptativo. Esto incluye balance-tlb y balance de carga en recepción (rlb) para tráfico IPv4. El balance de carga en recepción se logra por negociación ARP. El driver de bonding intercepta las respuestas ARP enviadas por el servidor y sobreescribe la dirección MAC origen con la dirección MAC única de uno de los esclavos en el bond, de forma que diferentes clientes usen diferentes direcciones MAC para dirigirse al server. 
\end{description}

\subsection {Detección de eventos}
Los modos que ofrecen tolerancia a fallos necesitan algún mecanismo para detectar eventos de caída de las interfaces de red (NIC) locales, los enlaces, o las NICs de los extremos opuestos de los vínculos. Ante la detección de un evento de fallo, el bond fuerza la conmutación a otra interfaz, que entonces se convierte en primaria. Esta acción se llama \emph{failover}.  

Para la detección de eventos hay dos opciones posibles.

\begin{enumerate}
		\item MII (Medium Independent Interface). Especificación que cumplen la mayoría de las NICs modernas, que presenta datos de link activo o inactivo en forma independiente de la implementación del medio conectado. Se debe especificar los parámetros \lstinline$bond-miimon$ (intervalo de revisión del estado del link), \lstinline$bond-downdelay$ (tiempo desde que se detecta fallo hasta que se da de baja la interfaz) y \lstinline$bond-updelay$ (tiempo para volver a poner en servicio la interfaz una vez que vuelve el link a estado activo). 
		\item ARP. Se establece por configuración una cantidad de direcciones IP de control en la red local, y el bond emite consultas ARP periódicas a estas direcciones. Cuando se deja de recibir respuesta por la interfaz activa durante una cantidad de tiempo, configurable, se considera que ha caído el enlace y se realiza el failover. Se deben configurar los parámetros \lstinline$bond-arp-interval$ (intervalo entre emisión de ARP) y \lstinline$bond-arp-ip-target$ (lista de IPs confiables).
\end{enumerate}
  
\subsection {Configuración en Debian}

\subsubsection {Configuración con MII}
\begin{lstlisting}
# /etc/network/interfaces
# The loopback network interface
auto lo
iface lo inet loopback
auto eth0
iface eth0 inet manual
	bond-master bond0
	bond-primary eth0 eth1

auto eth1
iface eth1 inet manual
	bond-master bond0
	bond-primary eth0 eth1

auto bond0
iface bond0 inet static
	address 192.168.1.15
	netmask 255.255.255.0
	network 192.168.1.0
	gateway 192.168.1.1
	bond-slaves eth0 eth1
	bond-mode active-backup
	bond-miimon 100
	bond-downdelay 200
	bond-updelay 200

\end{lstlisting}

\subsubsection {Configuración con ARP}

\begin{lstlisting}
# /etc/network/interfaces
# The loopback network interface
auto lo
iface lo inet loopback

# No se especifican las auto ethX

auto bond0
iface bond0 inet static
	address 10.1.1.1
	netmask 255.255.255.0
	network 10.1.1.0
	bond_primary eth0
	slaves eth0 eth1
	bond-mode active-backup
	bond-arp-interval 2000
	bond-arp-ip-target 10.1.1.2 10.1.1.3

\end{lstlisting}



\subsection{Temas de práctica}

La topología de la Fig. \ref{fig:bonding} comprende un nodo llamado switch, y tres nodos conectados a él. El switch tiene cinco interfaces, cada una conectada a un enlace diferente. Dos de los nodos tienen dos interfaces cada uno, y el tercero una sola. Cada interfaz de los nodos host1 a host3 está conectada a su propio enlace. Los enlaces se denominan, de izquierda a derecha en el diagrama, link1 a link5. 

\figura[10]{bonding}{Configuración del laboratorio de bonding}{bonding.png}

\recuadro{

¡Ver las Notas sobre el laboratorio en la sección más abajo!

}


\begin{enumerate}
	\item Configure en forma estática la red del nodo denominado switch implementando un bridge que esclavice a todas sus interfaces. Dé direcciones en el mismo dominio de broadcast a los demás nodos. Compruebe llegada por ping. Observe el tráfico que pasa por las interfaces del switch con el comando tcpdump -i ethX. El nodo virtual switch, ¿resulta un buen modelo de un switch de hardware? ¿Necesita tener una dirección IP? 
	\item En los nodos host1 y host2, Instale el módulo que permite esclavizar las interfaces con \code{dpkg -i /hostlab/ifenslave-...deb}.
	\item Configure el nodo host1 indicando en /etc/modules que debe ser cargado el módulo bonding al arranque. Configure la red del mismo nodo, ligando ambas interfaces mediante un bond en modo active-backup. ¿Qué capacidades tiene este modo?
	\item Mantenga un ping desde host3 a host1, observando el tráfico por las interfaces del switch. Simule la caída del vínculo con host1.  ¿Se interrumpe el ping? ¿Sigue pasando el tráfico por las mismas interfaces?
	\item Configure el nodo host2 del mismo modo que en el punto 3, pero con el bond en modo balance-rr. ¿Qué capacidades tiene este modo?
	\item Ejecute el mismo experimento del punto 4 pero desde host3 a host2. 
\end{enumerate}


\subsubsection{Notas sobre el laboratorio}
\begin{itemize}
	\item Es necesario instalar el paquete \code{ifenslave}. En nuestro laboratorio se ha provisto un ejemplar del archivo DEB correspondiente en el directorio \code{/hostlab}.
	\item Es necesario indicar el modo, técnica de detección de eventos, y parámetros, en el archivo \code{/etc/modules} (a pesar de indicarlo en \code{/etc/network/interfaces}), de la siguiente manera:
	\begin{lstlisting}
	alias bond0 bonding
	options bonding mode=1 miimon=100 downdelay=200 updelay=200
	\end{lstlisting}
(detección por MII) o bien:
	\begin{lstlisting}
	alias bond0 bonding
	options bonding mode=1 arp_interval=1000 arp_ip_target=10.1.1.1
	\end{lstlisting}
(detección por ARP).
	\item Al probar sistemas de Alta Disponibilidad, una parte muy importante es la inyección de fallas. La tecnología de virtualización utilizada para el laboratorio no permite replicar correctamente el evento de caída de interfaz, enlace o peer, cuando se usa la técnica de detección de fallas MII (la interfaz aparece siempre conectada). La mejor aproximación que hemos logrado a la simulación de la falla es cuando se elige detección de fallas por ARP.
	\item Para simular la caída de un vínculo se sugiere detener el proceso UML que simula el enlace. Logramos esto buscando entre todos los procesos \code{uml_switch} de la máquina host aquel relacionado con el nombre del enlace. Por ejemplo, para simular la caída del link1:
	\begin{lstlisting}
# ps f | grep uml_switch
23137 pts/1    S+     0:00  \_ grep uml_switch
...
19381 pts/1    S      0:00 /home/rii/netkit/bin/uml_switch -hub -unix /home/oso/.netkit/hubs/vhub_oso_link1.cnct
# kill -STOP 19381
\end{lstlisting}
	El proceso se hace continuar (se restablece el vínculo virtual) con \code{kill -CONT 19381}.
	\item Cada vez que hagamos un cambio de configuración será preferible, antes de hacer una nueva prueba, bajar y volver a levantar el laboratorio completo con \code{lhalt} y \code{lstart}. 
	\item La interfaz virtual bond puede monitorearse mirando el pseudo archivo correspondiente en el directorio /proc:
\begin{lstlisting}
# cat /proc/net/bonding/bond0 
Ethernet Channel Bonding Driver: v3.2.5 (March 21, 2008)

Bonding Mode: fault-tolerance (active-backup)
Primary Slave: eth0
Currently Active Slave: eth0
MII Status: up
MII Polling Interval (ms): 0
Up Delay (ms): 0
Down Delay (ms): 0
ARP Polling Interval (ms): 2000
ARP IP target/s (n.n.n.n form): 10.1.1.2, 10.1.1.3

Slave Interface: eth0
MII Status: up
Link Failure Count: 0
Permanent HW addr: a6:8a:d7:63:1e:49

Slave Interface: eth1
MII Status: up
Link Failure Count: 0
Permanent HW addr: e6:c9:c8:a3:eb:72
\end{lstlisting}
\end{itemize}



\subsection{Referencias}
\begin{itemize}
	\item \url{https://www.kernel.org/doc/Documentation/networking/bonding.txt}
	\item \url{http://www.linuxfoundation.org/collaborate/workgroups/networking/bonding}
	\item \url{http://www.cyberciti.biz/tips/debian-ubuntu-teaming-aggregating-multiple-network-connections.html}
\end{itemize}

\section{Balance de Carga}


Consideraremos en esta sección el problema de una organización con dos o más accesos a Internet a través de diferentes proveedores. Los enlaces extra se han contratado con la idea de seguir obteniendo acceso en caso de que uno de los proveedores falle. Sin embargo, resulta costoso diseñar una solución en modo \textit{activo-standby} (con sólo uno de los enlaces activo y sin aprovechar los demás), por lo cual en general esta solución debe incorporar alguna forma de distribución de tráfico por todos los enlaces, es decir, alguna forma de balance de carga. El ruteo default solamente permite indicar un camino de salida, de manera que se necesitará alguna técnica adicional para distribuir el tráfico.

Por la naturaleza de los protocolos de transporte de Internet, en general no será posible utilizar el total del ancho de banda agregado de todos los enlaces para una sola conexión, ya que segmentos TCP o UDP pertenecientes a la misma conversación con un servidor llegarán a destino provenientes de direcciones IP diferentes (aquellas que resulten del ruteo realizado por los diferentes proveedores). Este patrón de tráfico no puede formar parte de la misma conexión TCP o conversación UDP. Es decir, si se tienen dos accesos de 1Mbps cada uno, no es posible utilizar ambos enlaces agregados para efectuar una transferencia de datos a 2Mbps en una sola conexión. En cambio, sí es posible distribuir la totalidad de la carga entre todos los vínculos, destinando parte de las conexiones por uno u otro enlace. 

El escenario de múltiples accesos a Internet suele llamarse \textit{Split Access} (acceso dividido). Los dos problemas de Split Access son:

\begin{enumerate}
	\item Hacer que el tráfico ingresado al sistema desde Internet por una interfaz vuelva por la misma interfaz.
	\item Lograr distribución o balance de carga.
\end{enumerate}

\subsection{Herramientas}
Las dos principales herramientas de software para conseguir estos objetivos son el paquete \texttt{iproute2} y el comando \texttt{iptables}. Con el primero definiremos una estructura de ruteo especial para el caso de uso de Split Access y con el segundo manipularemos el tráfico para aplicar diferentes condiciones de ruteo.

\subsubsection{Paquete iproute2}
Paquete de configuración de red integral, que reemplaza a utilitarios como \texttt{ifconfig}, \texttt{route}, \texttt{netstat}. Permite configurar interfaces, información de ARP, ruteo por políticas, túneles, etc. Consiste de dos comandos: \texttt{ip}, que controla estado de interfaces y configuración de IPv4 e IPv6, y \texttt{tc}, que administra aspectos de calidad de servicio (\textit{Quality of Service}, QoS). Ejemplos en Cuadro \ref{tab:iproute2}.

\subsubsection{Comando iptables}
Es la interfaz de usuario al subsistema Netfilter del kernel, que permite modificar la forma como los paquetes IP atraviesan el kernel en su tránsito entre una interfaz de entrada, procesos locales, e interfaz de salida. Según el caso de cada paquete, aplica una serie de reglas impuestas por el usuario y agrupadas en diferentes \textbf{cadenas} (\texttt{PREROUTING, INPUT, OUTPUT, POSTROUTING, FORWARD}). Las reglas se almacenan en \textbf{tablas} que se aplican secuencialmente dentro de cada cadena. Las cadenas se disponen en diferentes lugares del recorrido del paquete por el kernel, y las tablas tienen una aplicación específica:

\begin{description}
	\item[Tabla \texttt{filter}] Aplica reglas de filtrado para implementar políticas de firewalling. 
	\item[Tabla \texttt{nat}] Ejecuta edición de direcciones. Por ejemplo, cuando se aplica NAT (\textit{Network Address Translation}), convierte la dirección de salida de un paquete a una dirección propia.
	\item[Tabla \texttt{mangle}] Permite efectuar cualquier modificación de cualquier zona del paquete, además de afectar datos que acompañan al paquete durante su viaje por el kernel. 
\end{description}

El comando \texttt{iptables} diferencia entre paquetes \textbf{destinados al host} (la dirección IP destino del paquete es alguna de las propias), \textbf{originados en el host} (la dirección origen es una dirección propia), o \textbf{en tránsito} (ni la dirección origen ni la dirección destino son propias). Este último caso es el que se presenta habitualmente en un router. 

\begin{description}
	\item[Ingreso] A todos los paquetes ingresantes, antes de que el proceso de ruteo determine si son o no dirigidos al host, se les aplican las reglas contenidas en las tablas \texttt{mangle} y \texttt{nat} de la cadena \texttt{PREROUTING}.
	\item[Egreso] Todos los paquetes que salen del host por cualquier motivo atraviesan las tablas \texttt{mangle} y \texttt{nat} de la cadena \texttt{POSTROUTING}. 
	\item[Paquetes destinados al host] Luego de determinarse que el paquete está destinado al host, éste atraviesa la cadena de \texttt{INPUT}, con las reglas que haya en la tabla \texttt{mangle} y la tabla \texttt{filter} de esa cadena, en ese orden (Fig. \ref{fig:iptables-input}). 
	\item[Paquetes originados en el host] Son producidos por procesos locales que emiten pedidos de servicio o responden a servicios solicitados. Atraviesan la cadena de \texttt{OUTPUT} (\texttt{mangle}, \texttt{nat} y \texttt{filter}, en ese orden) y luego se dirigen a la cadena de \texttt{POSTROUTING} (Fig. \ref{fig:iptables-output}). 
	\item[Paquetes que atraviesan el host] Se trata de paquetes que el router debe reenviar. Atraviesan las tablas \texttt{mangle} y \texttt{filter} de la cadena \texttt{FORWARD} y luego se dirigen a la cadena \texttt{POSTROUTING} (Fig. \ref{fig:iptables-forward}).
\end{description}


\tabla{iproute2}{Ejemplos de uso del comando ip}{l|l}{
\texttt{ip link list}					& Consultar interfaces \\
\texttt{ip address show}				& Consultar direcciones de interfaces \\
\texttt{ip neigh show}				& Tabla ARP\\
\texttt{ip route show}				& Rutas\\
\texttt{ip route list table main}		& Rutas, tabla principal\\
\texttt{ip route list table T}		& Rutas, tabla T\\
\texttt{ip route flush table main}	& Borrar rutas de tabla principal\\
\texttt{ip route flush cache}			& Borrar cache de rutas\\
\texttt{ip rule list}					& Consultar reglas de asignación de tablas\\
}
 
 
\figura[11]{iptables-input}{Recorrido de un paquete destinado al host}{iptables-input.pdf}
\figura[11]{iptables-output}{Recorrido de un paquete originado en el host}{iptables-output.pdf}
\figura[11]{iptables-forward}{Recorrido de un paquete reenviado por el host}{iptables-forward.pdf}

\subsection{Ruteo por políticas}
Un host determina por cuál de sus interfaces debe emitir un paquete mediante el proceso de ruteo, usando la información contenida en su tabla de ruteo o reenvío. Normalmente la tabla utilizada para el ruteo regular es única (la tabla \texttt{main}), pero apoyándose en el paquete \texttt{iproute2} y con la técnica de ruteo por políticas, se pueden utilizar diferentes tablas de ruteo. La decisión de qué tráfico utilizará cada tabla de ruteo se puede configurar en varias formas. Esta técnica permite asociar una tabla de ruteo distinta con cada acceso en el escenario de Split Access.

Primeramente es necesario definir nombres para las tablas. El paquete \texttt{iproute2} identifica las tablas con números, pero es más fácil para el administrador establecer unos nombres simbólicos en el archivo \texttt{/etc/iproute2/rt\_tables}. Simplemente se eligen números y nombres arbitrarios, que no colisionen con los ya reservados. Los nombres servirán posteriormente para referirse a las diferentes tablas de ruteo usando el comando \texttt{ip}, y pueden ser los nombres de los ISP o proveedores de acceso.

Una vez creados los nombres simbólicos, por cada acceso a Internet disponible es necesario identificar los siguientes datos.
\begin{itemize}
	\item Nombre de la tabla de ruteo ($P$).
	\item Dirección del router del proveedor ($R$).
	\item Dirección de red donde se ubica el router ($N$).
	\item Interfaz del router de la organización que comunica con el router del proveedor ($IF$).
	\item Dirección de dicha interfaz ($IP$).
\end{itemize}

Con estos datos, por cada acceso disponible:
\begin{enumerate}
	\item Se insertan las reglas de ruteo en su tabla correspondiente. 
\begin{lstlisting}
ip route add N dev IF src IP table P
ip route add default via R table P
\end{lstlisting}
	\item Se inserta en la tabla \texttt{main} la regla de ruteo al gateway del acceso.
	\begin{lstlisting}
ip route add N dev IF src IP
\end{lstlisting}
\end{enumerate} 

\subsection{Definición de políticas}
A continuación se establecen las reglas por las cuales uno u otro host cuyo tráfico se va a reenviar, utilizará una u otra tabla de ruteo. 

\subsubsection{Política según el origen}
La idea es simplemente separar los hosts de las redes locales en grupos, estáticamente, y derivar su tráfico por rutas diferentes y fijas. 

Por cada host o red con dirección IP que se quiere conducir por el acceso P:
\begin{lstlisting}
		ip rule add from IP table P 
\end{lstlisting}

\subsubsection{Política según el tráfico}

Utilizando \texttt{iptables} para manipular la tabla \texttt{mangle} de Netfilter, podemos crear \textbf{marcas} numéricas, que acompañan a los paquetes en su tránsito por el kernel. Estas marcas son reconocidas por el proceso de ruteo y sirven para definir la tabla de ruteo a utilizar para cada paquete.

Para el tráfico con origen $S$, destino $D$, al port $X$ (u otras condiciones que sea posible seleccionar mediante iptables), marcamos los paquetes en el ingreso con la marca numérica $Z$.
\begin{lstlisting}
iptables -t mangle -A PREROUTING -p tcp --dport X -s S -d D -j MARK --set-mark Z
\end{lstlisting}

Luego se establecen las reglas que derivan el tráfico con cada marca $Z$ a la tabla de ruteo $P$ que se desee:
\begin{lstlisting}
ip rule add fwmark Z table P
ip route flush cache
\end{lstlisting}

% 2. En round-robin
% Existe un módulo de Netfilter que emite circularmente un número módulo N para marcar paquetes. Para cada marca se agrega una regla que dirige los paquetes marcados a una tabla de ruteo determinada. Este módulo debería aplicarse a los paquetes SYN que originan una conexión. De lo contrario se descompone la conversación en los diferentes links, lo cual no sirve para conexiones outbound.
% Después:
% 
% ip rule add fwmark Z table Z
% ip route flush cache
% 

\subsection{Temas de práctica}
\subsubsection{Laboratorio virtual de Split Access}
La organización tiene dos redes locales o VLANs, LAN1 y LAN2, conectadas por un router. Se ha contratado acceso a Internet de dos proveedores diferentes. Ambos han provisto sus routers r1 y r2 con una dirección interna privada. El problema consiste en poder ofrecer a las redes locales de la organización acceso por uno u otro de los servicios, en la forma más flexible posible. 

\figura[10]{splitlab}{Esquema del laboratorio virtual de Split Access}{origen.png}

En el laboratorio se han fijado las direcciones que aparecen en el diagrama de Fig. \ref{fig:splitlab}. Todas las demás deben ser configuradas. Además se pide establecer ruteo por origen de modo que LAN1 acceda a Internet por r1 y LAN2 por r2.
\begin{enumerate}
	\item Comprobar que r1 y r2 tienen acceso a Internet. 
	\item Dar direcciones a las redes locales y establecer ruta por defecto para los hosts. 
	\item En r1 y r2, establecer rutas hacia las redes locales a través de \texttt{router}. 
	\item Comprobar que los hosts tienen ruta hacia las interfaces internas de r1 y r2. 
	\item En \texttt{router}, establecer una tabla de ruteo por cada proveedor en \texttt{/etc/iproute2/rt\_tables}. 
	\item Por cada proveedor, preparar su tabla de ruteo con regla default y especificacion de IP de salida. 
	\item Especificar en \texttt{router} un gateway default en la tabla main. 
	\item Agregar reglas de ruteo por origen estáticas por cada red local. 
	\item Desde los hosts, hacer ping a un host público de Internet. Ver con tcpdump en las interfaces internas de r1 y de r2 que router  aplica la política por origen correspondiente. 
	\item Incorporar excepciones (establecer que un determinado host utiliza un acceso diferente que el resto de la red a la que pertenece).
	\item En \texttt{router} reemplazar el ruteo simple por masquerading, comprobar que la política se sigue aplicando. 
	\item Modificar la configuración de las reglas de selección de tablas, aplicando una política que derive el tráfico de HTTP por el acceso 1 y todo el otro tráfico por el acceso 2.
	\item Establecer la excepción de que un determinado servidor HTTP se accederá siempre por el acceso 2. 
\end{enumerate}

\subsubsection{Notas del laboratorio}
Para dar acceso a Internet a las máquinas virtuales del laboratorio que funcionan como routers, es importante dar en la consola \textbf{del host} los comandos que enmascaran el tráfico de salida proveniente de esas máquinas virtuales.
\begin{lstlisting}
iptables -t nat -A POSTROUTING -s 172.16.100.0/24 -o eth0 -j MASQUERADE
iptables -A FORWARD -i eth0 -m state --state RELATED,ESTABLISHED -j ACCEPT
\end{lstlisting}

% 
% #### Multipathing
% Usando el keyword weight de ip route
% 
% #### Policy Routing con BGP
% 
% #### Usando TOS ("idea solamente, no testeado")
% Usando TC aplicado en la interfaz de entrada, y basándose en que ambos enlaces tengan características diferentes, marcar el TOS de los paquetes. Utilizarlo para rutear por política. Ventaja: aprovechar el análisis dinámico de tráfico hecho por TC, disciplinas de cola, etc. (se puede?).
% 
% 
% 
% 
% 
% #### 1. Respetar rutas de origen
% Se establecen dos tablas de ruteo por origen, una por cada proveedor. Aquí P1 y P2 son ISPs. 
% PX_NET es la red del ISP X
% IFX es la interfaz que da a PX_NET con dirección IPX.
% RX es un gateway default para cada tabla (next hop sobre la red del ISP X).
% 
% 	ip route add $P1_NET dev $IF1 src $IP1 table P1
% 	ip route add default via $R1 table P1
% 	
% 	ip route add $P2_NET dev $IF2 src $IP2 table P2
% 	ip route add default via $R2 table P2
% 	
% 	ip route add $P1_NET dev $IF1 src $IP1
% 	ip route add $P2_NET dev $IF2 src $IP2
% 	
% 	ip route add default via $P1
% 	# Reglas
% 	ip rule add from $IP1 table T1
% 	ip rule add from $IP2 table T2
% 
% 
% 
% #### Router
% /etc/iproute2/rt_tables <- tablas de proveedores
% 10	P1
% 20	P2
% 
% Por cada vínculo a un router del proveedor P, con dirección R sobre la red N, donde mi interfaz es IF y tiene dirección IP:
% ip route add N dev IF src IP table P
% ip route add default via R table P
% ip route add N dev IF src IP
% 
% Y ademas
% ip route add default via algúnR
% 
% Por cada host o red con direccion IP que se quiere conducir por prov P
% 
% ip rule add from IP table P
% 
% 
% 
% 
% #### 2. Balance de carga
% Para esto se utiliza multipathing, siendo posible ajustar la proporción de tráfico en cada canal mediante los modificadores "weight".
% 	ip route add default scope global nexthop via $P1 dev $IF1 weight 1 nexthop via $P2 dev $IF2 weight 1
% 
% 
% #### Links
% http://linux-ip.net/html/adv-multi-internet.html
% http://lartc.org/howto/lartc.netfilter.html
% http://lartc.org/howto/lartc.rpdb.multiple-links.html
% 





%----------- A N E X O S ---------
\newpage
\part {Anexos}
\appendix

\section{VLANs en Linux}

\subsection{Creación de VLAN}

\begin{lstlisting}
# ip link add link eth0 name eth0.VLAN2 type vlan id 2
\end{lstlisting}

La interfaz virtual eth0.VLAN2 se comportará como una interfaz Ethernet habitual. Todo el tráfico que sea dirigido a esta interfaz será enviado por la interfaz maestra pero llevará el tag con VLAN ID número 2. Este tráfico sólo puede ser aceptado por dispositivos con conocimiento del formato de trunking. 

\subsection{Consultar el VLAN ID de una interfaz}

\begin{lstlisting}
# ip -d link show eth0.VLAN2
\end{lstlisting}

\subsection{Dar una dirección IP}

\begin{lstlisting}
# ip addr add 192.168.100.1/24 brd 192.168.100.255 dev eth0.VLAN2
# ip link set dev eth0.VLAN2 up
\end{lstlisting}

\subsection {Desactivar administrativamente una interfaz}

\begin{lstlisting}
# ip link set dev eth0.VLAN2 down
\end{lstlisting}

\subsection {Eliminar una interfaz}

\begin{lstlisting}
# ip link delete eth0.VLAN2
\end{lstlisting}

\begin{comment}

Tomado de https://wiki.archlinux.org/index.php/VLAN
\subsubsection {Iniciar la interfaz al arranque del sistema}

\begin{lstlisting}
# netctl? # TODO
\end{lstlisting}



Troubleshooting



udev renames the virtual devices
An annoyance is that udev may try to rename virtual devices as they are added, thus ignoring the name configured for them (in this case eth0.100).
For instance, if the following commands are issued:
# ip link add link eth0 name eth0.100 type vlan id 100
# ip link show 
This could generate the following output:
1: lo: <LOOPBACK,UP,LOWER_UP> mtu 16436 qdisc noqueue state UNKNOWN 
    link/loopback 00:00:00:00:00:00 brd 00:00:00:00:00:00
2: eth0: <BROADCAST,MULTICAST,UP,LOWER_UP> mtu 1500 qdisc mq state UP qlen 1000
    link/ether aa:bb:cc:dd:ee:ff brd ff:ff:ff:ff:ff:ff
3: rename1@eth0: <BROADCAST,MULTICAST,UP,LOWER_UP> mtu 1500 qdisc noqueue state DOWN 
    link/ether aa:bb:cc:dd:ee:ff brd ff:ff:ff:ff:ff:ff
udev has ignored the configured virtual interface name eth0.100 and autonamed it rename1.
The solution is to edit /etc/udev/rules.d/network_persistent.rules and append DRIVERS=="?*" to the end of the physical interface's configuration line.
For example, for the interface aa:bb:cc:dd:ee:ff (eth0):
/etc/udev/rules.d/network_persistent.rules
SUBSYSTEM=="net", ATTR{address}=="aa:bb:cc:dd:ee:ff", NAME="eth0", DRIVERS=="?*"
A reboot should mean that VLANs configure correctly with the names assigned to them.


\end{comment}

\begin {comment}
\subsection{iptables.log}
\label{subsec:iptables.log}
\begin{lstlisting}
 Logged 539 packets on interface eth1
    From 200.42.136.212 - 3 packets to tcp(59351,59361)
\end{lstlisting}
\end{comment}


\section{Bridging en Linux}

\subsection{Interfaz para creación e inspección de bridges}
\begin{lstlisting}
# brctl 
Usage: brctl [commands]
commands:
	addbr     	<bridge>		add bridge
	delbr     	<bridge>		delete bridge
	addif     	<bridge> <device>	add interface to bridge
	delif     	<bridge> <device>	delete interface from bridge
	hairpin   	<bridge> <port> {on|off}	turn hairpin on/off
	setageing 	<bridge> <time>		set ageing time
	setbridgeprio	<bridge> <prio>		set bridge priority
	setfd     	<bridge> <time>		set bridge forward delay
	sethello  	<bridge> <time>		set hello time
	setmaxage 	<bridge> <time>		set max message age
	setpathcost	<bridge> <port> <cost>	set path cost
	setportprio	<bridge> <port> <prio>	set port priority
	show      	[ <bridge> ]		show a list of bridges
	showmacs  	<bridge>		show a list of mac addrs
	showstp   	<bridge>		show bridge stp info
	stp       	<bridge> {on|off}	turn stp on/off

\end{lstlisting}

\subsection {Configuración estática de bridges}
\label{subsec:staticbridge}
\subsubsection {Configuración en RedHat, CentOS, derivados}

Archivo /etc/sysconfig/network-scripts/ifcfg-br0
\begin{lstlisting}
DEVICE=br0
BOOTPROTO=none
ONBOOT=yes
TYPE=Bridge
IPADDR=192.168.0.1
NETMASK=255.255.255.0
NETWORK=192.168.0.0
BROADCAST=192.168.0.255
\end{lstlisting}

Archivo /etc/sysconfig/network-scripts/ifcfg-eth0
\begin{lstlisting}
TYPE=Ethernet
DEVICE=eth0
ONBOOT=yes
USERCTL=no
BRIDGE=br0
\end{lstlisting}

\subsubsection {Configuración en Debian, Ubuntu, derivados}

Archivo /etc/network/interfaces
\begin{lstlisting}
auto br0
iface br0 inet static
        address 10.10.0.15
        netmask 255.255.255.0
        gateway 10.10.0.1
        bridge_ports eth0 eth1
        up /usr/sbin/brctl stp br0 on
\end{lstlisting}        
        
\section {Ejemplos de configuración OpenVPN}

\subsection{Servidor}

\begin{lstlisting}
#################################################
# Sample OpenVPN 2.0 config file for            #
# multi-client server.                          #
#                                               #
# This file is for the server side              #
# of a many-clients <-> one-server              #
# OpenVPN configuration.                        #
#                                               #
# OpenVPN also supports                         #
# single-machine <-> single-machine             #
# configurations (See the Examples page         #
# on the web site for more info).               #
#                                               #
# This config should work on Windows            #
# or Linux/BSD systems.  Remember on            #
# Windows to quote pathnames and use            #
# double backslashes, e.g.:                     #
# "C:\\Program Files\\OpenVPN\\config\\foo.key" #
#                                               #
# Comments are preceded with '#' or ';'         #
#################################################

# Which local IP address should OpenVPN
# listen on? (optional)
;local a.b.c.d

# Which TCP/UDP port should OpenVPN listen on?
# If you want to run multiple OpenVPN instances
# on the same machine, use a different port
# number for each one.  You will need to
# open up this port on your firewall.
port 1194

# TCP or UDP server?
;proto tcp
proto udp

# "dev tun" will create a routed IP tunnel,
# "dev tap" will create an ethernet tunnel.
# Use "dev tap0" if you are ethernet bridging
# and have precreated a tap0 virtual interface
# and bridged it with your ethernet interface.
# If you want to control access policies
# over the VPN, you must create firewall
# rules for the the TUN/TAP interface.
# On non-Windows systems, you can give
# an explicit unit number, such as tun0.
# On Windows, use "dev-node" for this.
# On most systems, the VPN will not function
# unless you partially or fully disable
# the firewall for the TUN/TAP interface.
;dev tap
dev tun

# Windows needs the TAP-Win32 adapter name
# from the Network Connections panel if you
# have more than one.  On XP SP2 or higher,
# you may need to selectively disable the
# Windows firewall for the TAP adapter.
# Non-Windows systems usually don't need this.
;dev-node MyTap

# SSL/TLS root certificate (ca), certificate
# (cert), and private key (key).  Each client
# and the server must have their own cert and
# key file.  The server and all clients will
# use the same ca file.
#
# See the "easy-rsa" directory for a series
# of scripts for generating RSA certificates
# and private keys.  Remember to use
# a unique Common Name for the server
# and each of the client certificates.
#
# Any X509 key management system can be used.
# OpenVPN can also use a PKCS #12 formatted key file
# (see "pkcs12" directive in man page).
ca ca.crt
cert server.crt
key server.key  # This file should be kept secret

# Diffie hellman parameters.
# Generate your own with:
#   openssl dhparam -out dh1024.pem 1024
# Substitute 2048 for 1024 if you are using
# 2048 bit keys. 
dh dh1024.pem

# Configure server mode and supply a VPN subnet
# for OpenVPN to draw client addresses from.
# The server will take 10.8.0.1 for itself,
# the rest will be made available to clients.
# Each client will be able to reach the server
# on 10.8.0.1. Comment this line out if you are
# ethernet bridging. See the man page for more info.
server 10.8.0.0 255.255.255.0

# Maintain a record of client <-> virtual IP address
# associations in this file.  If OpenVPN goes down or
# is restarted, reconnecting clients can be assigned
# the same virtual IP address from the pool that was
# previously assigned.
ifconfig-pool-persist ipp.txt

# Configure server mode for ethernet bridging.
# You must first use your OS's bridging capability
# to bridge the TAP interface with the ethernet
# NIC interface.  Then you must manually set the
# IP/netmask on the bridge interface, here we
# assume 10.8.0.4/255.255.255.0.  Finally we
# must set aside an IP range in this subnet
# (start=10.8.0.50 end=10.8.0.100) to allocate
# to connecting clients.  Leave this line commented
# out unless you are ethernet bridging.
;server-bridge 10.8.0.4 255.255.255.0 10.8.0.50 10.8.0.100

# Configure server mode for ethernet bridging
# using a DHCP-proxy, where clients talk
# to the OpenVPN server-side DHCP server
# to receive their IP address allocation
# and DNS server addresses.  You must first use
# your OS's bridging capability to bridge the TAP
# interface with the ethernet NIC interface.
# Note: this mode only works on clients (such as
# Windows), where the client-side TAP adapter is
# bound to a DHCP client.
;server-bridge

# Push routes to the client to allow it
# to reach other private subnets behind
# the server.  Remember that these
# private subnets will also need
# to know to route the OpenVPN client
# address pool (10.8.0.0/255.255.255.0)
# back to the OpenVPN server.
;push "route 192.168.10.0 255.255.255.0"
;push "route 192.168.20.0 255.255.255.0"

# To assign specific IP addresses to specific
# clients or if a connecting client has a private
# subnet behind it that should also have VPN access,
# use the subdirectory "ccd" for client-specific
# configuration files (see man page for more info).

# EXAMPLE: Suppose the client
# having the certificate common name "Thelonious"
# also has a small subnet behind his connecting
# machine, such as 192.168.40.128/255.255.255.248.
# First, uncomment out these lines:
;client-config-dir ccd
;route 192.168.40.128 255.255.255.248
# Then create a file ccd/Thelonious with this line:
#   iroute 192.168.40.128 255.255.255.248
# This will allow Thelonious' private subnet to
# access the VPN.  This example will only work
# if you are routing, not bridging, i.e. you are
# using "dev tun" and "server" directives.

# EXAMPLE: Suppose you want to give
# Thelonious a fixed VPN IP address of 10.9.0.1.
# First uncomment out these lines:
;client-config-dir ccd
;route 10.9.0.0 255.255.255.252
# Then add this line to ccd/Thelonious:
#   ifconfig-push 10.9.0.1 10.9.0.2

# Suppose that you want to enable different
# firewall access policies for different groups
# of clients.  There are two methods:
# (1) Run multiple OpenVPN daemons, one for each
#     group, and firewall the TUN/TAP interface
#     for each group/daemon appropriately.
# (2) (Advanced) Create a script to dynamically
#     modify the firewall in response to access
#     from different clients.  See man
#     page for more info on learn-address script.
;learn-address ./script

# If enabled, this directive will configure
# all clients to redirect their default
# network gateway through the VPN, causing
# all IP traffic such as web browsing and
# and DNS lookups to go through the VPN
# (The OpenVPN server machine may need to NAT
# or bridge the TUN/TAP interface to the internet
# in order for this to work properly).
;push "redirect-gateway def1 bypass-dhcp"

# Certain Windows-specific network settings
# can be pushed to clients, such as DNS
# or WINS server addresses.  CAVEAT:
# http://openvpn.net/faq.html#dhcpcaveats
# The addresses below refer to the public
# DNS servers provided by opendns.com.
;push "dhcp-option DNS 208.67.222.222"
;push "dhcp-option DNS 208.67.220.220"

# Uncomment this directive to allow different
# clients to be able to "see" each other.
# By default, clients will only see the server.
# To force clients to only see the server, you
# will also need to appropriately firewall the
# server's TUN/TAP interface.
;client-to-client

# Uncomment this directive if multiple clients
# might connect with the same certificate/key
# files or common names.  This is recommended
# only for testing purposes.  For production use,
# each client should have its own certificate/key
# pair.
#
# IF YOU HAVE NOT GENERATED INDIVIDUAL
# CERTIFICATE/KEY PAIRS FOR EACH CLIENT,
# EACH HAVING ITS OWN UNIQUE "COMMON NAME",
# UNCOMMENT THIS LINE OUT.
;duplicate-cn

# The keepalive directive causes ping-like
# messages to be sent back and forth over
# the link so that each side knows when
# the other side has gone down.
# Ping every 10 seconds, assume that remote
# peer is down if no ping received during
# a 120 second time period.
keepalive 10 120

# For extra security beyond that provided
# by SSL/TLS, create an "HMAC firewall"
# to help block DoS attacks and UDP port flooding.
#
# Generate with:
#   openvpn --genkey --secret ta.key
#
# The server and each client must have
# a copy of this key.
# The second parameter should be '0'
# on the server and '1' on the clients.
;tls-auth ta.key 0 # This file is secret

# Select a cryptographic cipher.
# This config item must be copied to
# the client config file as well.
;cipher BF-CBC        # Blowfish (default)
;cipher AES-128-CBC   # AES
;cipher DES-EDE3-CBC  # Triple-DES

# Enable compression on the VPN link.
# If you enable it here, you must also
# enable it in the client config file.
comp-lzo

# The maximum number of concurrently connected
# clients we want to allow.
;max-clients 100

# It's a good idea to reduce the OpenVPN
# daemon's privileges after initialization.
#
# You can uncomment this out on
# non-Windows systems.
;user nobody
;group nogroup

# The persist options will try to avoid
# accessing certain resources on restart
# that may no longer be accessible because
# of the privilege downgrade.
persist-key
persist-tun

# Output a short status file showing
# current connections, truncated
# and rewritten every minute.
status openvpn-status.log

# By default, log messages will go to the syslog (or
# on Windows, if running as a service, they will go to
# the "\Program Files\OpenVPN\log" directory).
# Use log or log-append to override this default.
# "log" will truncate the log file on OpenVPN startup,
# while "log-append" will append to it.  Use one
# or the other (but not both).
;log         openvpn.log
;log-append  openvpn.log

# Set the appropriate level of log
# file verbosity.
#
# 0 is silent, except for fatal errors
# 4 is reasonable for general usage
# 5 and 6 can help to debug connection problems
# 9 is extremely verbose
verb 3

# Silence repeating messages.  At most 20
# sequential messages of the same message
# category will be output to the log.
;mute 20
\end{lstlisting}


\subsection{Cliente}

\begin{lstlisting}
##############################################
# Sample client-side OpenVPN 2.0 config file #
# for connecting to multi-client server.     #
#                                            #
# This configuration can be used by multiple #
# clients, however each client should have   #
# its own cert and key files.                #
#                                            #
# On Windows, you might want to rename this  #
# file so it has a .ovpn extension           #
##############################################

# Specify that we are a client and that we
# will be pulling certain config file directives
# from the server.
client

# Use the same setting as you are using on
# the server.
# On most systems, the VPN will not function
# unless you partially or fully disable
# the firewall for the TUN/TAP interface.
;dev tap
dev tun

# Windows needs the TAP-Win32 adapter name
# from the Network Connections panel
# if you have more than one.  On XP SP2,
# you may need to disable the firewall
# for the TAP adapter.
;dev-node MyTap

# Are we connecting to a TCP or
# UDP server?  Use the same setting as
# on the server.
;proto tcp
proto udp

# The hostname/IP and port of the server.
# You can have multiple remote entries
# to load balance between the servers.
remote my-server-1 1194
;remote my-server-2 1194

# Choose a random host from the remote
# list for load-balancing.  Otherwise
# try hosts in the order specified.
;remote-random

# Keep trying indefinitely to resolve the
# host name of the OpenVPN server.  Very useful
# on machines which are not permanently connected
# to the internet such as laptops.
resolv-retry infinite

# Most clients don't need to bind to
# a specific local port number.
nobind

# Downgrade privileges after initialization (non-Windows only)
;user nobody
;group nogroup

# Try to preserve some state across restarts.
persist-key
persist-tun

# If you are connecting through an
# HTTP proxy to reach the actual OpenVPN
# server, put the proxy server/IP and
# port number here.  See the man page
# if your proxy server requires
# authentication.
;http-proxy-retry # retry on connection failures
;http-proxy [proxy server] [proxy port #]

# Wireless networks often produce a lot
# of duplicate packets.  Set this flag
# to silence duplicate packet warnings.
;mute-replay-warnings

# SSL/TLS parms.
# See the server config file for more
# description.  It's best to use
# a separate .crt/.key file pair
# for each client.  A single ca
# file can be used for all clients.
ca ca.crt
cert client.crt
key client.key

# Verify server certificate by checking
# that the certicate has the nsCertType
# field set to "server".  This is an
# important precaution to protect against
# a potential attack discussed here:
#  http://openvpn.net/howto.html#mitm
#
# To use this feature, you will need to generate
# your server certificates with the nsCertType
# field set to "server".  The build-key-server
# script in the easy-rsa folder will do this.
ns-cert-type server

# If a tls-auth key is used on the server
# then every client must also have the key.
;tls-auth ta.key 1

# Select a cryptographic cipher.
# If the cipher option is used on the server
# then you must also specify it here.
;cipher x

# Enable compression on the VPN link.
# Don't enable this unless it is also
# enabled in the server config file.
comp-lzo

# Set log file verbosity.
verb 3

# Silence repeating messages
;mute 20

\end{lstlisting}

\section {Scripts para OpenVPN en modo bridge}
\label{sec:bridgeupdown}

\subsection{Archivo sample-scripts/bridge-start}

\begin{lstlisting}
#!/bin/bash

#################################
# Set up Ethernet bridge on Linux
# Requires: bridge-utils
#################################

# Define Bridge Interface
br="br0"

# Define list of TAP interfaces to be bridged,
# for example tap="tap0 tap1 tap2".
tap="tap0"

# Define physical ethernet interface to be bridged
# with TAP interface(s) above.
eth="eth0"
eth_ip="192.168.8.4"
eth_netmask="255.255.255.0"
eth_broadcast="192.168.8.255"

for t in $tap; do
    openvpn --mktun --dev $t
done

brctl addbr $br
brctl addif $br $eth

for t in $tap; do
    brctl addif $br $t
done

for t in $tap; do
    ifconfig $t 0.0.0.0 promisc up
done

ifconfig $eth 0.0.0.0 promisc up

ifconfig $br $eth_ip netmask $eth_netmask broadcast $eth_broadcast
\end{lstlisting}

\subsection{Archivo sample-scripts/bridge-stop}

\begin{lstlisting}
#!/bin/bash

####################################
# Tear Down Ethernet bridge on Linux
####################################

# Define Bridge Interface
br="br0"

# Define list of TAP interfaces to be bridged together
tap="tap0"

ifconfig $br down
brctl delbr $br

for t in $tap; do
    openvpn --rmtun --dev $t
done
\end{lstlisting}

%------------------------------------------------------------

\end{document}
