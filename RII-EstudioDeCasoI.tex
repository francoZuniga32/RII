
\section {Estudio de caso I}


Una organización desarrolla sus actividades en un campus con dos edificios principales, que alojan las oficinas y talleres. En la organización existen tres áreas principales: Operaciones, Comercialización, e Ingeniería. 

Cada área utiliza un servidor de base de datos principal que es propio del área. La organización cuenta además con un servidor de web y de correo electrónico, un servidor de archivos y un servidor de backups, los tres de uso general para las tres áreas. Se desea que todos los puestos de trabajo puedan acceder además a Internet. 

Una preocupación especial de la organización es darle protección a los servidores y puestos de trabajo de Ingeniería, que no deben ser accedidos desde las demás áreas.
 
La planta del campus y sus edificios, con los principales puntos donde se necesita conectividad, es como indica la Fig. \ref{fig:caso01}. En este diagrama se muestran, con diferentes colores, los puestos de trabajo de cada área. 


\figura[12]{caso01}{Conectando el campus de una organización}{caso01.jpg}


En base a esta información, indique:
\begin{itemize}
	\item Dónde situaría los servidores mencionados.
	\item Dónde ubicaría los elementos de conectividad (switches y routers).
	\item Qué cantidad de puertos debería tener cada uno de estos elementos.
	\item Con qué medios (cobre, fibra, inalámbricos) vincularía los clientes y los elementos de conectividad.
	\item Cómo distribuiría direcciones IP para los clientes y servidores. 
	\item Si utilizaría alguna arquitectura de VLANs, cuál, y por qué. En caso afirmativo, cómo relacionaría las VLANs entre sí, qué enlaces entre switches y routers deben ser de trunking y por qué.
	\item Si utilizaría alguna forma de redundancia, y en caso afirmativo, cuál debe ser el camino normal del tráfico y por qué.
\end{itemize}






